\section{Analítica Avanzada y Métricas (Fase 2)}

La Fase 2 del proyecto representó una evolución significativa de la plataforma, transformándola de un repositorio de almacenamiento y procesamiento masivo de datos hacia un sistema de analítica avanzada orientado a la generación de conocimiento estratégico. En esta etapa, el Grupo 3 se enfocó en la explotación del valor informacional contenido en los datos históricos de viajes de taxis de la ciudad de Nueva York.

Para cumplir este objetivo, se desarrolló un nuevo job de procesamiento distribuido en Apache Spark, denominado \texttt{analytics\_advanced.py}, encargado de calcular métricas analíticas de alto nivel. Paralelamente, se amplió la capa de servicio mediante la incorporación de una segunda versión de la API REST (\texttt{/api/v2}), diseñada específicamente para exponer indicadores consolidados y optimizados para consumo visual.

Esta fase consolidó el enfoque \textit{data-driven}, permitiendo que las decisiones analíticas se basen en datos procesados previamente, evitando cálculos en tiempo real y garantizando tiempos de respuesta óptimos.

\subsection{Definición de Métricas Clave}

Las métricas implementadas fueron seleccionadas bajo criterios de relevancia operativa, interpretabilidad y aplicabilidad en contextos reales de gestión del transporte urbano. En conjunto, estas métricas permiten analizar el comportamiento temporal, económico y geoespacial del sistema de movilidad.

Se definieron cinco dimensiones analíticas principales, descritas a continuación.

\subsubsection{1. Evolución Temporal de la Demanda}

\textbf{Objetivo:}  
Identificar patrones de comportamiento, estacionalidad y tendencias en la demanda del servicio a lo largo del tiempo.

\begin{itemize}
    \item \textbf{Método de cálculo:}  
    Agregación diaria del número total de viajes y cálculo del valor promedio de la tarifa por día, utilizando operaciones \texttt{groupBy} y funciones de agregación distribuidas de Spark.

    \item \textbf{Endpoint asociado:}  
    \texttt{/trips-over-time}

    \item \textbf{Valor analítico:}  
    Esta métrica permite detectar periodos atípicos de disminución o incremento de la demanda, como eventos extraordinarios, feriados, cambios regulatorios o crisis sanitarias, proporcionando una visión longitudinal del comportamiento del sistema.
\end{itemize}

\subsubsection{2. Preferencias de Pago del Usuario}

\textbf{Objetivo:}  
Analizar la adopción y evolución de los métodos de pago electrónicos frente al uso de efectivo.

\begin{itemize}
    \item \textbf{Método de cálculo:}  
    Conteo y normalización porcentual de los viajes agrupados por el atributo \texttt{payment\_type}. Los códigos numéricos originales del dataset fueron transformados en descripciones semánticas legibles (por ejemplo, 1 = Tarjeta de crédito, 2 = Efectivo).

    \item \textbf{Endpoint asociado:}  
    \texttt{/payment-stats}

    \item \textbf{Insight generado:}  
    Permite evaluar el nivel de digitalización del sistema de transporte y facilita la toma de decisiones relacionadas con la incorporación de nuevos mecanismos de cobro electrónico.
\end{itemize}

\subsubsection{3. Zonas de Alta Densidad de Viajes (Hotspots)}

\textbf{Objetivo:}  
Identificar espacialmente las áreas con mayor concentración de demanda.

\begin{itemize}
    \item \textbf{Método de cálculo:}  
    Ranking de las zonas de recogida (\texttt{PULocationID}) mediante un conteo descendente del volumen total de viajes, generando un listado \textit{Top N}.

    \item \textbf{Endpoint asociado:}  
    \texttt{/top-zones}

    \item \textbf{Utilidad operativa:}  
    Esta métrica permite optimizar la distribución de flotas, reducir tiempos de espera y mejorar la cobertura del servicio en zonas de alta actividad.
\end{itemize}

\subsubsection{4. Análisis del Comportamiento de Propinas}

\textbf{Objetivo:}  
Evaluar la relación existente entre la distancia recorrida y el porcentaje de propina otorgado al conductor.

\begin{itemize}
    \item \textbf{Método de cálculo:}  
    Segmentación de los viajes en intervalos de distancia (\textit{bins}), tales como 0--5 millas, 5--10 millas y rangos superiores, seguida del cálculo del porcentaje promedio de propina respecto al valor total del viaje.

    \item \textbf{Endpoint asociado:}  
    \texttt{/tip-analysis}

    \item \textbf{Insight analítico:}  
    Permite determinar si los viajes de mayor longitud influyen positivamente en la propensión del usuario a otorgar gratificaciones adicionales.
\end{itemize}

\subsubsection{5. Distribución de Distancias de Viaje}

\textbf{Objetivo:}  
Caracterizar la naturaleza del servicio ofrecido, diferenciando entre trayectos cortos, medianos y de largo recorrido.

\begin{itemize}
    \item \textbf{Método de cálculo:}  
    Construcción de un histograma de frecuencias de la variable \texttt{trip\_distance}, permitiendo visualizar la concentración de viajes por rango de distancia.

    \item \textbf{Endpoint asociado:}  
    \texttt{/distance-distribution}

    \item \textbf{Aplicación práctica:}  
    Facilita la comprensión del perfil promedio del viaje urbano y respalda decisiones relacionadas con tarifas, planificación vial y asignación de recursos.
\end{itemize}

\subsection{Endpoint Unificado de Monitoreo del Sistema}

Con el objetivo de optimizar el rendimiento del Dashboard principal y reducir la cantidad de solicitudes HTTP realizadas desde el cliente, se implementó el endpoint consolidado:

\begin{center}
\texttt{/system-stats}
\end{center}

Este endpoint se caracteriza por integrar información proveniente de tres fuentes heterogéneas dentro de una única respuesta JSON:

\begin{enumerate}
    \item \textbf{Infraestructura:}  
    Consulta en tiempo real a las métricas JMX del NameNode para obtener información sobre nodos activos, bloques replicados y posibles inconsistencias del clúster.

    \item \textbf{Almacenamiento:}  
    Cálculo del uso de disco correspondiente al directorio \texttt{/data/nyc}, incluyendo capacidad total, espacio utilizado y disponibilidad restante.

    \item \textbf{Indicadores de negocio:}  
    Consolidación del número total histórico de viajes procesados por la plataforma.
\end{enumerate}

La integración de estas métricas técnicas y analíticas proporciona una visión holística del sistema, permitiendo evaluar simultáneamente el estado de la infraestructura y el valor del conjunto de datos procesados.

Esta fase consolida a la plataforma desarrollada por el Grupo 3 como una solución de analítica Big Data completa, capaz de transformar grandes volúmenes de información en conocimiento accionable.
